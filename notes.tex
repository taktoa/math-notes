\documentclass[]{article}

\usepackage{screenread}

\usepackage{lmodern}
\usepackage{amssymb,amsmath}
\usepackage{ifxetex,ifluatex}
\usepackage{fixltx2e}
\usepackage{xcolor}

\usepackage[T1]{fontenc}
\usepackage[utf8]{inputenc}

\definecolor{LinkColor}{HTML}{006090}

\IfFileExists{upquote.sty}{\usepackage{upquote}}{}
\IfFileExists{microtype.sty}{\usepackage[]{microtype}%
  \UseMicrotypeSet[protrusion]{basicmath}}{}
\PassOptionsToPackage{hyphens}{url}
\usepackage[unicode=true]{hyperref}
\hypersetup{
  unicode    = true,
  pdfborder  = {0 0 0},
  breaklinks = true,
  colorlinks = true,
  allcolors  = LinkColor
}
\urlstyle{same}
\IfFileExists{parskip.sty}{\usepackage{parskip}}{%
  \setlength{\parindent}{0pt}\setlength{\parskip}{6pt plus 2pt minus 1pt}}
\setlength{\emergencystretch}{3em}
\providecommand{\tightlist}{\setlength{\itemsep}{0pt}\setlength{\parskip}{0pt}}
\setcounter{secnumdepth}{0}

\ifx\paragraph\undefined\else
\let\oldparagraph\paragraph
\renewcommand{\paragraph}[1]{\oldparagraph{#1}\mbox{}}
\fi
\ifx\subparagraph\undefined\else
\let\oldsubparagraph\subparagraph
\renewcommand{\subparagraph}[1]{\oldsubparagraph{#1}\mbox{}}
\fi

\makeatletter
\def\fps@figure{htbp}
\makeatother

\usepackage{mathtools}
\usepackage{amsmath}

\DeclarePairedDelimiter{\abs}{\lvert}{\rvert}

\definecolor{AnswerColor}{HTML}{664411}

\newcommand{\answer}[1]{\color{AnswerColor} #1}
\newenvironment{answereq*}[0]{%
  \color{AnswerColor}\begin{equation*}}{\end{equation*}}

\newcommand{\atsign}[0]{@}

\usepackage[at]{easylist}

\newcommand{\taylor}[3]{%
  \ensuremath{\mathcal{T}_{{#1} \, \mathrm{at} \, {#2}}^{#3}}}

\usepackage{stmaryrd}

\newcommand{\semantic}[1]{\ensuremath{\llbracket {#1} \rrbracket}}

\newcommand{\textbs}[1]{{\sffamily\fontseries{sbc}\selectfont #1}}

\newcommand{\mathbs}[1]{\ensuremath{\text{\textbs{#1}}}}
\renewcommand{\mathtt}[1]{\ensuremath{\texttt{#1}}}

\newcommand{\mrs}[1]{\ensuremath{\mathnormal{#1}}} % Reset font to normal
\newcommand{\mbf}[1]{\ensuremath{\mathbf{#1}}}     % Boldface
\newcommand{\mbs}[1]{\ensuremath{\mathbs{#1}}}     % Bold + sans-serif
\newcommand{\mbb}[1]{\ensuremath{\mathbb{#1}}}     % Blackboard bold
\newcommand{\mtt}[1]{\ensuremath{\mathtt{#1}}}     % Teletype
\newcommand{\mrm}[1]{\ensuremath{\mathrm{#1}}}     % Serif ("roman")
\newcommand{\msf}[1]{\ensuremath{\mathsf{#1}}}     % Sans-serif
\newcommand{\msc}[1]{\ensuremath{\mathsc{#1}}}     % Small-caps
\newcommand{\mcl}[1]{\ensuremath{\mathcal{#1}}}    % Calligraphic
\newcommand{\msr}[1]{\ensuremath{\mathscr{#1}}}    % Script
\newcommand{\mfr}[1]{\ensuremath{\mathfrak{#1}}}   % Fraktur

\newcommand{\norm}[1]{\ensuremath{\lVert{} {#1} \rVert{}}}

\newcommand{\rmand}{\ensuremath{\mathrel{\mathrm{and}}}}

\newcommand{\tens}[0]{\otimes}
\newcommand{\comp}[0]{\circ}

\newcommand{\join}[0]{\ensuremath{\mathrel{\sqcup}}}
\newcommand{\meet}[0]{\ensuremath{\mathrel{\sqcap}}}
\newcommand{\bigjoin}[0]{\ensuremath{{\sqcup}\,}}
\newcommand{\bigmeet}[0]{\ensuremath{{\sqcap}\,}}
\newcommand{\poset}[0]{\ensuremath{\mathrel{\sqsubseteq}}}
\newcommand{\tesop}[0]{\ensuremath{\mathrel{\sqsupseteq}}}

\newcommand{\catset}[0]{\mathsf{Set}}
\newcommand{\catmon}[0]{\mathsf{Mon}}
\newcommand{\catgrp}[0]{\mathsf{Grp}}
\newcommand{\catab}[0]{\mathsf{Ab}}
\newcommand{\cat}[1]{\mathsf{#1}}

\newcommand{\opcat}[1]{{#1}^{\mathrm{op}}}

\DeclareMathOperator{\interior}{Int}
\DeclareMathOperator{\closure}{Cl}

\renewcommand{\complement}[1]{{{#1}^{c}}}

\DeclareMathOperator{\argmin}{arg\,min}

\newcommand{\define}[1]{\textsf{#1}}

\begin{document}

\section{Probability Theory}

\begin{easylist}[itemize]
@ These notes are primarily based on the following sources:
@@ {%
  \href{https://web.math.princeton.edu/~nelson/books/rept.pdf}{%
    \textit{Radically Elementary Probability Theory}}
  by Edward Nelson
}
@@ {%
  The notes I wrote when I took UIUC IE 300: Analysis of Data.
}
@ {%
  A \define{finite probability space} is a tuple
  $(\Omega \in \catset, \mathrm{pr} : \Omega \to \mbb{R})$ such
  that $\sum \{\mrm{pr}(\omega) \mid \omega \in \Omega\} = 1$
  and $\forall \omega \in \Omega ~.~ \mathrm{pr}(\omega) > 0$.
}
@ {%
  A \define{random variable} on $\Omega$ is a function $X : \Omega \to \mbb{R}$.
}
@@ {%
  The \define{expectation} of a random variable $X$, denoted $\mbb{E}(X)$, is
  defined by
  \begin{equation*}
  \mbb{E}(X)
  = \sum \{ X(\omega) \cdot \mrm{pr}(\omega) \mid \omega \in \Omega \}
  \end{equation*}
  \vspace{-2em}
}
@ {%
  An \define{event} is a subset $A \subseteq \Omega$ of the set underlying a
  finite probability space.
}
@@ {%
  The \define{probability} of an event is defined by
  \begin{equation*}
  \mbb{P}(A)
  = \sum \{ \mrm{pr}(\omega) \mid \omega \in A \}
  \end{equation*}
  \vspace{-2em}
}
@@ {%
  For any event $A$, the \define{indicator function} of $A$, denoted $\chi_A$,
  is a random variable defined by
  \begin{equation*}
  \chi_A(\omega)
  = \left\{
  \begin{array}{lr}
    1 & \text{when } \omega \in    A \\
    0 & \text{when } \omega \notin A
  \end{array}
  \right\}
  \end{equation*}
  We can think of the probability of an event as being the expectation of the
  indicator function for that event: $\mbb{P}(A) = \mbb{E}(\chi_A)$.
}
@@ {%
  The \define{complementary event} for any event $A$, denoted $\complement{A}$,
  is defined by $\complement{A} = \Omega \setminus A$.
}
@ {%
  The set $\Omega \to \mbb{R}$ of all random variables on $\Omega$ is an
  $n$-dimensional vector space, where $n = \abs{\Omega}$.
}
\end{easylist}

\section{Abstract Algebra}

\subsection{Ideals}

\begin{easylist}[enumerate]
@ {%
  An \define{ideal} $I$ of a ring $R$ should be thought of as a particular kind
  of subset of a ring.
}
@ {%
  Specifically, for some ring $(R, +, {}\cdot{})$, a set $I$ is a
  \define{two-sided ideal} of $R$ if it satisfies two conditions:
}
@@ \textbf{Subgroup}: $(I, +)$ is a subgroup of $(R, +)$
@@ {%
  \textbf{Absorption}: For any $r \in R$ and any $i \in I$, $r \cdot i$ and
  $i \cdot r$ are both elements of $I$.
}
@ {%
  We can generalize this to the notion of a \define{left ideal} and a
  \define{right ideal}. A left ideal only requires $r \cdot i \in I$, while a
  right ideal only requires $i \cdot r \in I$.
}
@ {%
  The canonical example of an ideal is that of the even integers $2\mbb{Z}$,
  which are an ideal in $\mbb{Z}$.
}
@@ {%
  To see why this is, note that addition is closed over $2\mbb{Z}$, and that
  the additive inverse of an even integer is also even. Thus, $(2\mbb{Z}, +)$
  is a subgroup of $(\mbb{Z}, +)$. The other factor is that multiplying any
  number by an even number gives you another even number, so the absorption
  condition is satisfied.
}
@ {%
  More generally, for any integer $n$ in $\mbb{Z}$, $n\mbb{Z}$ (i.e.: the set
  $\{n \cdot x \mid x \in \mbb{Z}\}$) is an ideal in $\mbb{Z}$.
}
@ {%
  In some sense, ideals generalize the idea of the set of all values
  divisible by a given value.
}
@ Other examples:
@@ {%
  For any ring $R$, $R$ is trivially an ideal of $R$.
  This is called the \define{unit ideal} of $R$.
}
@@ {%
  For any ring $R$, $\{0_R\}$ is an ideal of $R$.
  This is called the \define{zero ideal} of $R$.
}
@@ {%
  Denote the ring of all univariate polynomials with real coefficients
  by $\mbb{R}[x]$. Then the set of all such polynomials divisible by $x^2 + 1$
  is an ideal in $\mbb{R}[x]$.
}
@ Some types of ideals:
@@ {%
  A \define{proper ideal}  is one that is not the unit ideal.
  A \define{nonzero ideal} is one that is not the zero ideal.
}
@@ {%
  The \define{maximal ideal} of a ring is the largest possible proper ideal
  for that ring.
}
@@ {%
  The \define{minimal ideal} of a ring is the smallest possible nonzero ideal
  for that ring.
}
@@ {%
  A \define{prime ideal} is an ideal $I$ such that for any $(a, b) \in R^2$,
  $a \cdot b \in I$ implies $a \in I$ or $b \in I$.
}
@@ {%
  A \define{radical} or \define{semiprime ideal} is an ideal $I$ such that
  for any $a \in R$, $a^n \in I$ implies $a \in I$.
}
@@ A \define{principal ideal} is an ideal with one generator.
@ {%
  The \define{quotient of a ring $R$ by an ideal $I$} is
  $(R / \{(a, b) \in R^2 \mid (a - b) \in I\}, +, {}\cdot{})$.
}
@ Products and sums
@@ {%
  Any two ideals $I$ and $J$ have a sum, defined as
  $I + J = \{a + b \mid a \in I \land b \in J\}$.
}
@@ {%
  Any two ideals $I$ and $J$ have a product, defined as
  \begin{equation*}
  I \times J = \{
    \psi(0) + \cdots + \psi(n)
    \mid   n \in \mbb{N}
    \rmand \phi \in \mbb{N} \to I \times J
    \rmand \psi = \pi \comp \phi
  \}
  \end{equation*}
  where $\pi : R^2 \to R$ is defined as $\pi(a, b) = a \cdot b$
}
@@ {%
  Note that $(I \cup J) \subseteq (I + J)$
  and $(I \times J) \subseteq (I \cap J)$.
}
@ The set of ideals of a ring $R$ is denoted $\mbb{I}_R$.
@ {%
  For convenience, we will define a function
  $\Phi : (A^2 \to A) \times A \to \mbb{P}(A) \to A$ by:
  \begin{equation*}
    {\Phi(f, e)(\varnothing) = e}
    ~ ~ \rmand ~ ~
    {\Phi(f, e)(\{x\} \cup X) = f(x, \Phi(f, e)(X))}
  \end{equation*}
  where $x \in A$ and $X \subseteq \mbb{P}(A)$.
  This is known as the \define{fold} of a binary operator $f$ with a unit $e$.
}
@ {%
  $(\mbb{I}_R, \subseteq, \Phi(+, \{0_R\}), \Phi(\cap, R), \{0_R\}, R)$
  is a complete modular lattice.
}
\end{easylist}

\section{Locale Theory}

\begin{easylist}[itemize]
@ These notes are primarily based on the following sources:
@@ {%
  \href{https://dl.acm.org/citation.cfm?id=64996}{%
    \textit{Topology via Logic}}
  by Steven Vickers
}
@ For some set $A$ and relation $R \subseteq A \times A$:
@@ $R$ is \define{reflexive} iff for all $x \in A$, $(x, x) \in R$.
@@ {%
  $R$ is \define{transitive} iff for all $(x, y, z) \in A^3$ such that
  $(x, y) \in R$ and $(y, z) \in R$, we have $(x, z) \in R$.
}
@@ {%
  $R$ is \define{symmetric} iff for all $(x, y) \in A^2$,
  $(x, y) \in R \iff (y, x) \in R$.
}
@@ {%
  $R$ is \define{anti-symmetric} iff for all $(x, y) \in A^2$ such that
  $(x, y) \in R$ and $(y, x) \in R$, we have $x = y$.
}
@ {%
  A \define{preorder} is a set $P$ equipped with a relation $(\poset)$ that is
  both reflexive and transitive.
}
@@ {%
  Alternatively, a preorder can be thought of as a category in which for any
  pair of objects $(X, Y)$, we have $\abs{X \to Y} \le 1$. For this reason,
  preorders are sometimes called \define{thin categories}.
}
@@ {%
  The \define{opposite preorder} for a preorder $P = (X, \poset)$ is defined as
  $\opcat{P} = (X, \tesop)$.
}
@@ {%
  A function $f : (P, \poset_P) \to (Q, \poset_Q)$ is \define{monotone} iff
  $\forall (x, y) \in P^2 ~ . ~ x \poset_P y \implies f(x) \poset_Q f(y)$.
}
@ A \define{poset} is a preorder in which $(\poset)$ is anti-symmetric.
@@ {%
  Alternatively, a poset can be thought of as a category in which for any pair
  of objects $(X, Y)$, we have $\abs{(X \to Y) \cup (Y \to X)} \le 1$.
}
@@ {%
  Every preorder $P$ gives rise to a poset $P / (\equiv)$, where
  $a \equiv b \iff (a \poset b) \land (b \poset a)$.
}
@@ {%
  In a given poset $P$, with $X \subseteq P$ and $a \in P$, $a$ is a
  \define{lower bound} (resp. \define{upper bound}) for $X$ iff for any
  $x \in X$, $a \poset x$ (resp. $a \tesop x$).
}
@@ {%
  A lower bound $a$ of $X$ is a \define{meet} if it is greater than or equal to
  any other lower bound.
}
@@ {%
  A \define{join} is dual to a meet; if $a$ is a join for $X$ in $P$, then
  $a$ is a meet for $X$ in $\opcat{P}$.
}
@ {%
  A \define{pseudolattice} is a poset in which every nonempty finite subset has
  a meet and a join.
}
@ {%
  A \define{meet-semilattice} (resp. \define{join-semilattice}) is a poset in
  which every finite subset has a meet (resp. join).
}
@ {%
  A \define{lattice} is a poset that is both a meet-semilattice and a
  join-semilattice. Alternatively, a lattice is a pseudolattice with empty meets
  and joins (which correspond to unique maximal and minimal elements).
}
@ {%
  \textbf{Theorem}:
  A poset $P$ is a pseudolattice if it has binary meets and joins.
}
@ {%
  \textbf{Corollary}:
  If $P$ also has empty meets and joins, it is a lattice.
}
@ {%
  As a result, we will heretofore denote all finite meets and joins by infix
  binary operators $(\meet)$ and $(\join)$.
}
@ {%
  Infinite meets and joins will look like, e.g.: $\bigmeet_i(p_i)$ for a
  sequence $p : \mbb{N} \to P$ or $\bigmeet(X)$ for some $X \subseteq P$.
}
@ {%
  For lattices $L_1$ and $L_2$, a function $f : L_1 \to L_2$ is a
  \define{lattice homomorphism} iff $f(a \meet_1 b) = f(a) \meet_2 f(b)$
  and $f(a \join_1 b) = f(a) \join_2 f(b)$ for all $(a, b) \in L_1^2$.
}
@ Note that the set of all pseudolattices is closed under $(\opcat{-})$.
@ {%
  A poset $P$ is a \define{frame} iff every subset has join, every finite subset
  has a meet, and binary meets distribute over joins.
}
@@ In general, if $F$ is a frame, $\opcat{F}$ is not a frame.
@@ {%
  One notable exception is the \define{powerset frame} on a set $X$,
  $\mbb{P}(X)$.
}
@@ {%
  $\mbf{1} = (\{\star\}, \o)$, the \define{inconsistent frame}, so named
  because it has $\bot = \top$.
}
@@ $\mbf{2} = (\{\bot,\top\}, \{(\bot, \top)\})$, the \define{Sierpinski frame}.
@ {%
  A \define{topology} $\Omega$ on a set $X$ is a frame such that $\Omega$ is a
  subframe of $\mbb{P}(X)$.
}
@ A \define{topological space} is a set $X$ equipped with a topology $\Omega$.
@ The elements of $\Omega$ are known as the \define{open subsets} of $X$.
@ For a poset $P$ where $x \in P$ and $S \subseteq P$:
@@ {%
  ${\uparrow}(x) = \{y \in P \mid x \poset y\}$
  is the \define{upper closure} of $x$.
}
@@ {%
  ${\uparrow}(S) = \{y \in P \mid (\exists x \in S ~ . ~ x \poset y)\}$
  is the \define{upper closure} of $S$.
}
@@ {%
  The \define{lower closure} of $x$ and $S$, denoted by ${\downarrow}(-)$,
  is just the upper closure in $\opcat{P}$.
}
@ Examples of topologies:
@@ The \define{discrete topology} is $\Omega = \mbb{P}(X)$.
@@ The \define{indiscrete topology} is $\Omega = \o$.
@@ {%
  The \define{Alexandrov topology} on a poset $P$ is defined as
  $\Omega = \{{\uparrow}(x) \mid x \in P\}$.
}
@ {%
  We will heretofore denote the topology associated with any given topological
  space $X$ by $\Omega_X$.
}
@ {%
  The \define{interior} of a subset $S$ of a topological space $X$ is
  $\interior(S) = \cup \{A \in \Omega_X \mid A \subseteq S\}$
}
@@ Note that $\interior(S)$ is always the largest open set contained in $S$.
@ A subset $F \subseteq X$ is \define{closed} iff its complement is open.
@ A subset is \define{clopen} iff it is both open and closed.
@ {%
  The \define{topological closure} of a subset $S \subseteq X$ is
  $\closure(S) = \complement{(\interior(\complement{S}))}$
}
\end{easylist}

\end{document}
